\documentclass[11pt,a4paper]{article}
\usepackage[utf8]{inputenc}
\usepackage[margin=1in]{geometry}
\usepackage{listings}
\usepackage{xcolor}
\usepackage{hyperref}

\title{WiFi Data Import Guide\\Multi-Sensor Network}
\author{Multi-Sensor Network Documentation}
\date{\today}

\lstset{
    language=Python,
    basicstyle=\ttfamily\small,
    keywordstyle=\color{blue},
    stringstyle=\color{red},
    commentstyle=\color{gray},
    numbers=left,
    numberstyle=\tiny\color{gray},
    breaklines=true,
    frame=single,
    backgroundcolor=\color{gray!10}
}

\begin{document}

\maketitle

\section{Overview}

This document explains how to send sensor data from ESP32 devices (or any WiFi-enabled device) to the multi-sensor network dashboard via HTTP POST requests.

\section{API Endpoint}

The main data ingestion endpoint is:
\begin{verbatim}
POST /api/ingest
\end{verbatim}

Full URL format:
\begin{verbatim}
https://your-vercel-app.vercel.app/api/ingest
\end{verbatim}

\section{Data Format}

The API expects a JSON payload with the following structure:

\begin{lstlisting}[language=JSON]
{
    "temperature": 23.5,
    "humidity": 45.2,
    "co2": 420
}
\end{lstlisting}

\subsection{Required Fields}
\begin{itemize}
    \item \texttt{co2}: Number (required) - CO$_2$ level in ppm
    \item \texttt{temperature}: Number (optional) - Temperature in degrees Celsius
    \item \texttt{humidity}: Number (optional) - Humidity percentage
\end{itemize}

\section{Example Implementations}

\subsection{ESP32 Arduino Code}

\begin{lstlisting}[language=C++]
#include <WiFi.h>
#include <HTTPClient.h>
#include <ArduinoJson.h>

const char* ssid = "YOUR_WIFI_SSID";
const char* password = "YOUR_WIFI_PASSWORD";
const char* serverURL = "https://your-app.vercel.app/api/ingest";

void setup() {
    Serial.begin(115200);
    WiFi.begin(ssid, password);
    
    while (WiFi.status() != WL_CONNECTED) {
        delay(500);
        Serial.print(".");
    }
    Serial.println("WiFi connected");
}

void sendSensorData(float temp, float humidity, int co2) {
    if (WiFi.status() == WL_CONNECTED) {
        HTTPClient http;
        http.begin(serverURL);
        http.addHeader("Content-Type", "application/json");
        
        StaticJsonDocument<200> doc;
        doc["temperature"] = temp;
        doc["humidity"] = humidity;
        doc["co2"] = co2;
        
        String jsonString;
        serializeJson(doc, jsonString);
        
        int httpResponseCode = http.POST(jsonString);
        
        if (httpResponseCode > 0) {
            Serial.print("HTTP Response code: ");
            Serial.println(httpResponseCode);
        } else {
            Serial.print("Error code: ");
            Serial.println(httpResponseCode);
        }
        
        http.end();
    }
}

void loop() {
    float temp = readTemperature();
    float humidity = readHumidity();
    int co2 = readCO2();
    
    sendSensorData(temp, humidity, co2);
    delay(5000);
}
\end{lstlisting}

\subsection{Python Example}

\begin{lstlisting}[language=Python]
import requests
import json
import time

API_URL = "https://your-app.vercel.app/api/ingest"

def send_sensor_data(temperature, humidity, co2):
    payload = {
        "temperature": temperature,
        "humidity": humidity,
        "co2": co2
    }
    
    headers = {
        "Content-Type": "application/json"
    }
    
    try:
        response = requests.post(API_URL, 
                                json=payload, 
                                headers=headers)
        response.raise_for_status()
        print(f"Success: {response.json()}")
        return True
    except requests.exceptions.RequestException as e:
        print(f"Error: {e}")
        return False

# Example usage
if __name__ == "__main__":
    while True:
        send_sensor_data(
            temperature=23.5,
            humidity=45.2,
            co2=420
        )
        time.sleep(5)
\end{lstlisting}

\subsection{cURL Command}

For testing from command line:

\begin{lstlisting}[language=bash]
curl -X POST https://your-app.vercel.app/api/ingest \
  -H "Content-Type: application/json" \
  -d '{
    "temperature": 23.5,
    "humidity": 45.2,
    "co2": 420
  }'
\end{lstlisting}

\section{Response Format}

\subsection{Success Response}
\begin{lstlisting}[language=JSON]
{
    "ok": true
}
\end{lstlisting}

\subsection{Error Responses}

Invalid payload (missing or invalid co2):
\begin{lstlisting}[language=JSON]
{
    "ok": false,
    "error": "Invalid payload"
}
\end{lstlisting}

HTTP Status Codes:
\begin{itemize}
    \item \texttt{200}: Success
    \item \texttt{400}: Invalid payload
    \item \texttt{405}: Method not allowed (must use POST)
\end{itemize}

\section{Important Notes}

\begin{itemize}
    \item The \texttt{co2} field is \textbf{required} and must be a number
    \item Use HTTPS for secure connections
    \item The API supports CORS, so it can be called from web browsers
    \item For production, implement error handling and retry logic
    \item Consider rate limiting to avoid overwhelming the server
    \item The dashboard polls for new data every second
\end{itemize}

\section{Troubleshooting}

\begin{itemize}
    \item \textbf{Connection timeout}: Check WiFi connection and server URL
    \item \textbf{400 Bad Request}: Verify JSON format and required fields
    \item \textbf{SSL errors}: Ensure device supports HTTPS/TLS
    \item \textbf{No data on dashboard}: Check that data is being sent successfully (check response code)
\end{itemize}

\section{Additional Resources}

\begin{itemize}
    \item API Status endpoint: \texttt{GET /api/status}
    \item Dashboard: \texttt{https://your-app.vercel.app/}
    \item Repository: \url{https://github.com/IsaakBtue/multi-sensor-vercel}
\end{itemize}

\end{document}

